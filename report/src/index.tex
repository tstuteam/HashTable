
\section*{Постановка задачи}
Необходимо разработать универсальный класс для хранения данных в виде хеш-таблицы.

Универсальный класс хеш-таблица может иметь тип $<K, V>$, где $K$ - ключ, а $V$ - значение.

Класс должен быть сериализуемым, то есть его можно было бы конвертировать в \textit{JSON} или \textit{XML} формат, а так же десериализуемым, чтобы можно из файла создавать структуру данных во время исполнения программы.

Проект должен иметь тесты, графический интерфейс, консольное приложение и саму библиотеку.
\addcontentsline{toc}{section}{Постановка задачи}

\newpage
\section{Введение}

В компьютерных науках хеш-таблица (также можно использовать слово хэш) (hash-map) представляет собой структуру данных, которая реализует абстрактный тип данных ассоциативного массива, структуру, которого является сопоставлением ключей с значением.
Хеш-таблица использует хэш-функцию для вычисления индекса, также называемого хэш-кодом, в массиве сегментов (buckets) или слотов, из которых можно получить желаемое значение. 
Во время поиска ключ хэшируется, и результирующий хэш указывает, где хранится соответствующее значение.

В идеале хэш-функция будет назначать каждый ключ уникальному сегменту, но в большинстве конструкций хэш-таблиц используется несовершенная хэш-функция, которая может привести к коллизии хешей, которое происходит, когда хэш-функция генерирует один и тот же индекс для более чем одного ключа. 
Такие коллизии обычно каким-то образом разрешаются.

В хэш-таблице с некими размерами средняя стоимость (количество инструкций) для каждого поиска не зависит от количества элементов, хранящихся в таблице.
Многие конструкции хэш–таблиц также допускают произвольные вставки и удаление пар ключ-значение при постоянной (амортизированной) средней стоимости за операцию.

Хеширование - это пример компромисса между пространством и временем. 
Если память бесконечна, все ключи можно использовать непосредственно в качестве индекса для определения его значения с помощью одного прямого доступа к месту памяти элемента. 
С другой стороны, если время бесконечно, значения могут храниться без учета их ключей, и двоичный поиск или линейный поиск могут быть использованы для извлечения элемента.

Во многих ситуациях хэш-таблицы оказываются в среднем более эффективными, чем деревья поиска или любая другая структура поиска по таблицам. 
По этой причине они широко используются во многих видах компьютерного программного обеспечения, особенно для ассоциативных массивов, индексации баз данных, кэшей и множеств.

\newpage
\section*{Заключение}
Делаем заключение. Типа работу сделали, можно доделать то-то сё-то.
\addcontentsline{toc}{section}{Заключение}